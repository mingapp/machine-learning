\documentclass{article}
\usepackage{xeCJK}
\setmainfont{Lucida Console}
\setCJKmainfont{Microsoft YaHei}

\usepackage
	[colorlinks,linkcolor=blue]%,urlcolor=magenta!40!yellow]
	{hyperref}

\author{Manfred}
\title{Machine Learning Public Information}

\begin{document}
\maketitle
\newpage

\tableofcontents
\newpage

\listoftables
\newpage

\listoffigures
\newpage

\section{机器学习}
\href{http://zh.wikipedia.org/wiki/%E6%9C%BA%E5%99%A8%E5%AD%A6%E4%B9%A0}{机器学习@wikipedia}

机器学习是近20多年兴起的一门多领域交叉学科,涉及概率论、统计学、逼近论、凸分析、算法复杂度理论等多门学科。机器学习理论主要是设计和分析一些让计算机可以自动“学习”的算法。机器学习算法是一类从数据中自动分析获得规律,并利用规律对未知数据进行预测的算法。因为学习算法中涉及了大量的统计学理论,机器学习与统计推断学联系尤为密切,也被称为统计学习理论。算法设计方面,机器学习理论关注可以实现的,行之有效的学习算法。很多推论问题属于无程序可循难度,所以部分的机器学习研究是开发容易处理的近似算法。

机器学习已经有了十分广泛的应用,例如:数据挖掘、计算机视觉、自然语言处理、生物特征识别、搜索引擎、医学诊断、检测信用卡欺诈、证券市场分析、DNA序列测序、语音和手写识别、战略游戏和机器人运用。

目录

    1 定义
    2 机器学习相关条目
    3 参看
    4 参考书目
    5 外部链接

\subsection{定义}

机器学习有下面几种定义: “机器学习是一门人工智能的科学,该领域的主要研究对象是人工智能,特别是如何在经验学习中改善具体算法的性能”。 “机器学习是对能通过经验自动改进的计算机算法的研究”。 “机器学习是用数据或以往的经验,以此优化计算机程序的性能标准。” 一种经常引用的英文定义是:A computer program is said to learn from experience E with respect to some class of tasks T and performance measure P, if its performance at tasks in T, as measured by P, improves with experience E.
\subsection{机器学习相关条目}

机器学习可以分成下面几种类别:

    监督学习从给定的训练数据集中学习出一个函数,当新的数据到来时,可以根据这个函数预测结果。监督学习的训练集要求是包括输入和输出,也可以说是特征和目标。训练集中的目标是由人标注的。常见的监督学习算法包括回归分析和统计分类。
    无监督学习与监督学习相比,训练集没有人为标注的结果。常见的无监督学习算法有聚类。
    半监督学习介于监督学习与无监督学习之间。
    增强学习通过观察来学习做成如何的动作。每个动作都会对环境有所影响,学习对象根据观察到的周围环境的反馈来做出判断。

具体的机器学习算法有:

    构造条件概率:回归分析和统计分类
        人工神经网络
        决策树(Decision tree)
        高斯过程回归
        线性判别分析
        最近邻居法
        感知器
        径向基函数核
        支持向量机
    通过再生模型构造概率密度函数(Probability density function):
        最大期望算法(Expectation-maximization algorithm)
        graphical model:包括贝叶斯网和Markov随机场
        Generative Topographic Mapping
    近似推断技术:
        马尔可夫链(Markov chain)蒙特卡罗方法
        变分法
    最优化(Optimization):大多数以上方法,直接或者间接使用最优化算法。

\subsection{参看}
\begin{itemize}
\item    人工智能
\item    计算智能
\item    数据挖掘(Data mining)
\item    模式识别(Pattern recognition)
\item    机器学习方面重要出版物(计算机科学)
\item    机器学习方面重要出版物(统计学)
\item    自主控制机器人
\item    归纳逻辑编程
\item    决策树
\item    神经网络
\item    强化学习
\item    贝叶斯学习
\item    最近邻居法
\item    计算学习理论
\end{itemize}
\subsection{参考书目}
\begin{itemize}
\item    Bishop, C. M. (1995). 模式识别神经网络,牛津大学出版社。ISBN 0-19-853864-2
\item    Bishop, C. M. (2006). 模式识别与机器学习,Springer。ISBN 978-0-387-31073-2
\item    Richard O. Duda, Peter E. Hart, David G. Stork (2001) 模式分类(第2版), Wiley, New York, ISBN 0-471-05669-3.
\item    MacKay, D. J. C. (2003). 信息理论,推理和学习算法,剑桥大学出版社ISBN 0-521-64298-1
\item    Mitchell, T. (1997). 机器学习, McGraw Hill. ISBN 0-07-042807-7
\item    Sholom Weiss, Casimir Kulikowski (1991). Computer Systems That Learn, Morgan Kaufmann. ISBN 1-55860-065-5
\end{itemize}

\end{document}
